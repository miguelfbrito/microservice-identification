\chapter{Introdução}

    % ser explicito à cerca do ganho de popularidade da cultura devops e das infraestruturas da cloud
    Nos últimos anos, houve um despoletar da discussão e utilização do paradigma arquitetural de micro-serviços resultante da crescente necessidade em melhorar a modularidade e flexibilidade dos sistemas de \textit{software}, acompanhada do ganho de popularidade da cultura  de \textit{DevOps} e das infraestruturas na \textit{cloud} \citep{most_prominent_areas_of_research_ms_hamzehloui}. 
    
    Uma vez que o paradigma de micro-serviços é relativamente recente, a literatura é reduzida, havendo uma grande necessidade de investigação para procurar entender e formalizar processos associados aos micro-serviços \citep{most_prominent_areas_of_research_ms_hamzehloui}. 
    % dar contexto sobre de onde aparecem as migrações e descrever o interesse generalizado de realizar as migrações
    Dadas as vantagens proclamadas por este paradigma comparativamente a arquiteturas monolíticas, há de uma forma generalizada um interesse em migrar aplicações que seguem uma arquitetura monolítica para micro-serviços.
    
    Num estudo realizado por \cite{fritzsch19_migration_in_industry} com o objetivo de entender como são realizados os processos de migração na indústria, os especialistas que se propunham a realizar a decomposição de vários sistemas monolíticos para micro-serviços, revelaram total desconhecimento da possibilidade de recorrer a métodos para auxiliar e automatizar o processo, incluindo alguns \textit{anti-patterns} nas suas propostas de decomposição dos sistemas \citep{pahl16_ms_systematic_mapping}. Para lá dos eventuais problemas introduzidos pela limitada experiência dos especialistas \citep{pahl16_ms_systematic_mapping}, este é um processo custoso e demorado, daí a necessidade em automatizá-lo.
    % TODO: adicionar transição 
    
    Esta dissertação tem como intenção explorar os principais desafios relativos à transição de um sistema monolítico para uma arquitetura orientada a micro-serviços e propor uma metodologia e uma ferramenta que auxilie nesse processo, reduzindo o tempo e investimento necessário nos processos de \textit{refactoring}.
    
    
    
\section{Motivação}

    Um dos principais problemas no decorrer do desenvolvimento de uma aplicação de larga escala, passa pela crescente dificuldade na adição de nova funcionalidade e manutenção da atual dado o aumento de complexidade do projeto em questão \citep{buildingmaintainablesoftware2016}. Deste incremento de complexidade, tipicamente funcionalidades semelhantes começam a espalhar-se por diferentes partes da aplicação, tornando a identificação e correção de \textit{bugs} mais difícil. De forma a lidar com este problema, em ambientes monolíticos, criam-se abstrações para garantir uma maior coesão entre funcionalidades semelhantes, procurando seguir o princípio da responsabilidade única - juntar coisas que se alteram pela mesma razão, e separar as que se alteram por razões diferentes \citep{newman2015microservices}. No entanto, as próprias abstrações com a sua evolução aumentam no seu grau de complexidade, altura em que as vantagens da arquitetura monolítica serão inferior às suas desvantagens \citep{monolithtomicroserviceschen17}. 
    
    Da dificuldade em gerir a complexidade inerente da evolução dos projetos de grande escala, dada a sua complexidade, maior dificuldade em realizar alterações, manutenção e \textit{deploys} surgem os micro-serviços: pequenos serviços autónomos e coesos, com o objetivo de garantirem e manterem o princípio da responsabilidade única. O principal objetivo de cada serviço é realizar um pequeno conjunto de tarefas relativas a uma parte específica do domínio e realizá-las bem. Os micro-serviços permitem uma maior heterogeneidade de tecnologias, sendo possível cada micro-serviço ser programado numa linguagem diferente desde que haja uma tecnologia comum para comunicação. Permitem também um outro conjunto de vantagens: desde uma melhor resiliência, dado o pequeno tamanho dos serviços, uma escalabilidade mais eficiente, sendo os recursos focados na funcionalidade que realmente necessita, facilidade de desenvolvimento uma vez que uma equipa poderá estar focada apenas no seu micro-serviço não tendo que se preocupar com o panorama geral de funcionamento da aplicação. De uma forma geral, os micro-serviços dão resposta a várias das dificuldades que aparecem ao longo da evolução de uma aplicação monolítica.
    
    % TODO : Dar um fim a este parágrafo, transição mais suave

    O processo de transição de uma aplicação monolítica para uma arquitetura de micro-serviços mais modular é um processo custoso, trabalhoso e cheio de desafios únicos a cada sistema \citep{migratinglegacykazanavicius19}, que poderá obrigar à paragem do desenvolvimento do produto atual. Este processo encontra-se limitado pelo conhecimento e experiência do especialista a realizá-lo \citep{pahl16_ms_systematic_mapping}, sendo comum a introdução de \textit{anti-patterns} na decomposição do sistema \citep{fritzsch19_migration_in_industry}.
    
    Desta forma, uma metodologia e uma ferramenta que auxilie na identificação e extração de módulos coesos será essencial em diminuir o custo e tempo investido nesta transição, quer por identificar de forma automática possíveis micro-serviços, quer por realizar a maioria do \textit{refactoring} para que a proposta extraída seja funcional. 
    
    % este processo está muito dependente da experiência do EXPERT
    % poderá obrigar à realização de várias iterações
    % muitos dos arquitetos não conhecem sequer que se pode sequer automatizar o processo

    No entanto, será sempre importante que haja um papel ativo das entidades responsáveis na avaliação dos micro-serviços propostos, tendo em mente o modelo de negócio e requisitos funcionais.
    







