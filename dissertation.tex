\documentclass[oneside,11pt,a4paper,footinclude=true,headinclude=true,cleardoublepage=empty]{scrbook}
% example for dissertation.sty
  % Replace oneside by twoside if you are printing your thesis on both sides
  % of the paper, leave as is for single sided prints or for viewing on screen.
  %twoside,

\usepackage{dissertation}
\usepackage{indentfirst}
\usepackage{pgfgantt}
\usepackage{floatrow}
\usepackage[graphicx]{realboxes}
\usepackage{rotating}
\usepackage[portuguese]{babel}


% ACRONYMS -----------------------------------------------------

%import the necessary package with some options
\usepackage[acronym,nonumberlist,nomain]{glossaries}
\usepackage{import}

%enable the following to avoid links from the acronym usage to the list
%\glsdisablehyper

%displays the first use of an acronym in italic
\defglsdisplayfirst[\acronymtype]{\emph{#1#4}}

%the style of the Glossary
\glossarystyle{listgroup}

% set the name for the acronym entries page
\renewcommand{\glossaryname}{Acronyms}

%this shall be the last thing in the acronym configuration!!
\makeglossaries


% here are the acronym entries
\newacronym{mei}{MEI}{Mestrado em Engenharia Informática}
\newacronym{di}{DI}{Departamento de Informática}
\newacronym{um}{UM}{Universidade do Minho}

\newacronym{qos}{QoS}{Quality of Service}
\newacronym{soa}{SOA}{Service Oriented Architecture}

% these could go in an acronyms.tex file, and loaded with:
% \loadglsentries[\acronymtype]{Parts/Definitions/acronyms}
% when using this, you may want to remove 'nomain' from the package options

%% **MORE INFO** %%

%to add the acronyms list add the following where you want to print it:
%\printglossary[type=\acronymtype]
%\clearpage
%\thispagestyle{empty}

%to use an acronym:
%\gls{qps}

% compile the thesis in command line with the following command sequence:
% pdlatex dissertation.tex
% makeglossaries dissertation
% bibtex dissertation
% pdlatex dissertation.tex
% pdlatex dissertation.tex

% ----------------------------------------------------------------

% Title

% \titleA{Reengenharia de aplicações \textit{web} monolíticas }
\titleA{Extração de micro-serviços de aplicações}
\titleB{\textit{web} monolíticas}
% \titleB{Second Part of Title} % (if any)
% \subtitleA{First Part of Subtitle}
% \subtitleB{Second part of Subtitle} % (if any)

% Author
\author{Miguel António Ferrão Brito}

% Supervisor(s)
\supervisor{Prof. Dr. Jácome Cunha}
\cosupervisor{Prof. Dr. João Saraiva}

% University (uncomment if you need to change default values)
% \def\school{Escola de Engenharia}
% \def\department{Departamento de Inform\'{a}tica}
% \def\university{Universidade do Minho}
% \def\masterdegree{Computer Science}

% Date
\date{\myear} % change to text if date is not today

% Keywords
%\keywords{master thesis}

% Glossaries & Acronyms
%\makeglossaries  %  either use this ...
%\makeindex	   % ... or this

% Define Acronyms
%!TEX root = ../dissertation.tex

%\newacronym{}{}{}
\newacronym{REST}{REST}{Representational State Transfer}
\newacronym{DFD}{DFD}{Data-Flow Diagram}
\newacronym{CRUD}{CRUD}{Create, Read, Update and Delete}
\newacronym{API}{API}{Application Programming Interface}
\newacronym{HTTP}{HTTP}{Hypertext Transfer Protocol}
\newacronym{AMQP}{AMPQ}{Advanced Message Queuing Protocol}
\newacronym{SDG}{SDG}{System Dependence Graph}

%\glsaddall[types={\acronymtype}]



\ummetadata % add metadata to the document (author, publisher, ...)

\begin{document}
	% Cover page ---------------------------------------
	\umfrontcover	
	\umtitlepage
	
	%\import{sec/}{0-acknowledgements.tex}
	%\import{sec/}{1-abstract.tex}
	%\cleardoublepage
	%\import{sec/}{2-resumo.tex}
	
	% Summary Lists ------------------------------------
	\tableofcontents
	\listoffigures
%	\listoftables
	%\lstlistoflistings
	%\listofabbreviations
% 	\printglossary[type=\acronymtype]
	\clearpage
	\thispagestyle{empty}

	
	\pagenumbering{arabic}

	\import{sec/}{4-introduction.tex}
	
	\import{sec/}{5-context.tex}
        
	\import{sec/}{6-state_of_the_art.tex}

	\import{sec/}{7-problem.tex}



	% CHAPTER - Contribution -------------------------
%	\chapter{Development}
		
%	\section{Decisions}
%    \section{Implementation}
%    \section{Outcomes}
%    Main result(s) and their scientific evidence
%	\section{Summary}


	% CHAPTER - Application -------------------------
%	\chapter{Case Studies / Experiments}
%		Application of main result (examples and case studies)
%	\section{Experiment setup}
%    \section{Results}
%    \section{Discussion}
%	\section{Summary}
%
%	% CHAPTER - Conclusion/Future Work --------------
%	\chapter{Conclusion}
%		Conclusions and future work.
%	\section{Conclusions}
%	\section{Prospect for future work}
			
	\bookmarksetup{startatroot} % Ends last part.
	\addtocontents{toc}{\bigskip} % Making the table of contents look good.
	%\cleardoublepage

	%- Bibliography (needs bibtex) -%
	\bibliography{dissertation}

	% Index of terms (needs  makeindex) -------------
	%\printindex
	
	
	% APPENDIX --------------------------------------
%	\umappendix{Appendix}
	
	% Add appendix chapters
%	\chapter{Support material}
%	Auxiliary results which are not main-stream; or

	%\chapter{Details of results}
%	Details of results whose length would compromise readability of main text; or

	%\chapter{Listings}
%	Specifications and Code Listings: should this be the case; or

	%\chapter{Tooling}
%	Tooling: Should this be the case.

	%Anyone using \Latex\ should consider having a look at \TUG,
	%the \tug{\TeX\ Users Group}


	% Back Cover -------------------------------------------
%	\umbackcover{
%	NB: place here information about funding, FCT project, etc in which the work is framed. Leave empty otherwise.
%	}


\end{document}
